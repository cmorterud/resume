% LaTeX resume using res.cls
\documentclass[overlapped]{res}
\usepackage{parcolumns}
\usepackage{hyperref}
\usepackage{enumitem}
\setlength{\textwidth}{7.4in} % set width of text portion
\addtolength{\oddsidemargin}{-0.85in}
\addtolength{\topmargin}{-0.9in}
\addtolength{\textheight}{1.75in}
\addtolength{\sectionwidth}{-0.2in}
\addtolength{\sectionskip}{-0.05in}



\begin{document}
\begin{resume}


% Center the name over the entire width of resume:
\moveleft.5\hoffset\centerline{\Large\bf Cody Morterud}
% Draw a horizontal line the whole width of resume:
 \moveleft.5\hoffset\vbox{\hrule width 0.9\paperwidth height 1pt}\smallskip
% address begins here
% Again, the address lines must be centered over entire width of resume:
\moveleft.5\hoffset\centerline{ 
    cmorteru@umich.edu 
    $\cdot$ \href{http://www.codymorterud.com}{www.codymorterud.com} $\cdot$ (248) 481-0355
    $\cdot$ 784 Hidden Creek Drive
    $\cdot$ South Lyon, MI 48178
}

\section{EDUCATION} 
\textbf{University of Michigan, Ann Arbor, MI}
\hfill April 2019\\ 
                 {\sl Bachelor of Science in Engineering,} Computer Science,
                 {\sl Minor, Mathematics}\hfill 3.42/4.00 GPA\\
                 {\sl Admitted to Engineering Honors Program}\\
                 Relevant Coursework: Operating Systems, System Design of a Search Engine,
                 Algorithms \& Data Structures,
                 Web Systems, Networking, Machine Learning, Cryptography,
                 Computer Security
%\textbf{Washtenaw Community College, Ann Arbor, MI} 
%\hfill August 2015 -  July 2016 \\
%4.00/4.00 GPA

\section{EXPERIENCE}
\textbf{Quicken Loans, Detroit, MI} \hfill May 2018 - August 2018 \\
{\sl Software Developer Intern}
\begin{itemize}  \itemsep -2pt %reduce space between items

    \item Addressed market research for a new internal intern networking service
    and won hackathon award
    by implementing, presenting, and demoing 
    a brand new cloud-based internal
    networking platform with a team of interns via
    AWS Lambda, DynamoDB, API Gateway, S3 and Cloudfront.

    \item Automated InRule test suites by developing, testing, and publishing
    company-wide Nuget package to add
    necessary functionality to InRule testing environment.

    \item Promoted continuous integration standards by setting up build templates to build,
    test, apply secrets, conduct code analysis,
    and check quality gates before release.

    \item Met continuous deployment quality gates by writing unit tests
    with NUnit and MOQ to ensure high code coverage.

    \item Standardized data models across services
    and promoted best practices
    by publishing Nuget class library package.

    \item Provided visibility into DynamoDB for business users
    by designing Splunk dashboard with use of
    custom queries.

    \item Identified bottlenecks in services by adding application-level
    metrics logging to services via InfluxDB,
    and set up Grafana dashboards to visually view the metrics.

    \item Ensured service level agreements are met by setting up email alerting 
    for production errors and high service latency.

    \item Developed with C\texttt{\#} .NET Core and .NET Framework
    for applications currently in production.

    \item Met business requirements for applications by learning
    and developing in domain specific language.

    \item Tracked and measured feature and story progress through
    Agile framework ceremonies and scrums.

\end{itemize}

\textbf{The Boyle Lab, Ann Arbor, MI}
\hfill October 2016 - May 2018 \\
{\sl Research Assistant}
    \begin{itemize}  \itemsep -2pt %reduce space between items
    \item Decreased data processing time approximately 80\% 
    and increased research throughput
    by translating and refactoring bioinformatics algorithm 
    from Perl to C\texttt{++}.
    
    \item Identified bottlenecks in algorithm by analyzing stack trace samples with \textit{perf}.
    
    \item Satisfied design requirements by integrating third-party programs 
    and libraries into project.
    
    \item Maximized performance by using 
    Standard Template Library containers and algorithms,
    and tracked changes via version control systems.

    \item Ensured accurate implementation and results by working closely
     with benefactors and stakeholders.
\end{itemize}

\section{PROJECT EXPERIENCE}

\begin{itemize}[label={}]  \itemsep -2pt %reduce space between items
    \item \textbf{C\texttt{++}, Make, GDB:}
        Implemented, tested, and debugged internet crawler
        for search engine capstone to retrieve HTML documents 
        to construct an inverted index, facilitating user search queries.
    \item \textbf{C\texttt{\#} .NET, WPF:}
        Designed open source message encryption interface using
        Windows Presentation Forms to allow non-technical users
        to communicate securely using AES-128 and AES-256 in CBC mode
        according to NIST standards.
    \item \textbf{Python, Flask, React, SQLite:}
        Implemented Instagram clone with basic functionality 
        to gain familiarity with the Flask and React frameworks, 
        developing web pages via templating, 
        and database construction with SQLite.
    \item \textbf{Python:}
        Developed MapReduce master and worker server system 
        with REST API interface to serve concurrent MapReduce jobs
        and gain experience programming asynchronously.
    \item \textbf{C\texttt{++}, Make, GDB:} Devised asynchronous, 
        network file system with user and request
        validation for CRUD operations to study file system implementation
        and socket programming.
     \item \textbf{C\texttt{++}, Make, GDB:}
         Implemented thread library 
         to gain familiarity with concurrent
         programming utilities such as threads, mutexes, and condition variables.
    %\item \textbf{Python, MapReduce, Flask:}
    %Built Wikipedia search engine with use of Hadoop 
    %to construct an inverted index via MapReduce
    %to calculate cosine similarity scores 
    %to facilitate user search queries.
    % \item Developed single-level pager for a simulated system to efficiently
    % allocate virtual pages to memory
    % and reduce page faults and learn about memory management unit implementation.
\end{itemize}

\section{COMPUTER SKILLS}
\bigskip
\begin{parcolumns}[rulebetween,colwidths={5=2cm,1=2cm}]{6}
    \colchunk[1]{\textbf{Languages}\\C/C\texttt{++}\\Python\\C\texttt{\#} .NET\\\LaTeX}
    \colchunk[2]{\textbf{IDE}\\Visual Studio\\Visual Studio Code\\VIM\\Xcode}
    \colchunk[3]{\textbf{Technologies}\\Git\\Bash\\Splunk\\Grafana\\Nuget}
    \colchunk[4]{\\MOQ\\TFS\\AWS Lambda\\AWS API Gateway\\AWS DynamoDB}
    \colchunk[5]{\\InRule\\Flask\\Perf\\MapReduce\\Postman}
    \colchunk[6]{\\WPF\\GDB\\Valgrind\\Make\\PyTorch}
\end{parcolumns}

\end{resume}
\end{document}

